%------------------------------------------------------------------------------
%Description:  LaTeX Thesis Template preamble
%Author:       silas.kalmbach@ilek.uni-stuttgart.de
%Created:      2021-03-10
%------------------------------------------------------------------------------

% ============= Randabstände =============

\usepackage[left= 2.5cm, right = 2cm, top=2.5cm]{geometry}


% ============= Dokumentinformationen =============

\usepackage[ 
hidelinks
]{hyperref} 
\hypersetup{ 
	pdfauthor={\student}, 
	pdftitle={\titel},
	pdfsubject={\arbeit}, 
	pdfkeywords={}
}


% ============= Farbschema =============

\usepackage{color}      %Verwendung von Farben
\usepackage{xcolor}		%Definition eigener Farben

% Uni Stuttgart Colors
\definecolor{uniSlightblue}{RGB}{0,190,255}
\definecolor{uniSdarkblue}{RGB}{0,65,145}
\definecolor{uniSdarkgrey}{RGB}{62, 68, 76}
\definecolor{uniSgreyblue}{RGB}{125,198,234}
\definecolor{uniSgrey}{RGB}{159,153,152}

% Uni Stuttgart monochrom greyscale
\definecolor{uniSblack}{RGB}{0,0,0}
\definecolor{uniSdarkgrey}{RGB}{62,68,76}
\definecolor{uniSgrey}{RGB}{159,153,152}
\definecolor{uniSlightgrey}{RGB}{185,186,187}


% ============= Standart Packages und Einstellungen =============

\usepackage{ifluatex}
\ifluatex
	\usepackage[utf8]{luainputenc} %Schriftkodierung LuaLaTeX
\else
	\usepackage[utf8]{inputenc} %Schriftkodierung PDFLaTeX
\fi

\usepackage[english, ngerman]{babel} %Sprachen
\usepackage{graphicx} %Einbinden von Bildern
\usepackage{amsfonts} % zusätzliche Schriftzeichen der American Mathematical Society
\usepackage{amsmath,amssymb} % zusätzliche Schriftzeichen der American Mathematical Society
\usepackage{typearea} %Aufteilung von Rändern und Textbereich
\usepackage{nicefrac}% weitere Alternative mit mehreren Zeichen für Brüche
\usepackage{pdfpages} %Einfügen von PDFs
\usepackage{appendix} %Anhang
\usepackage{fvextra} %automatic line breaking and improved math mode
\usepackage{csquotes} %inline and display quotations
\usepackage[german, nameinlink]{cleveref} % Mehr Möglichkeiten zum referenzieren
\usepackage{gensymb} %Symbole: \degree \celsius \perthousand \ohm \micro
\tolerance = 400  % möglicher Abstand zwischen den Wörtern erhöhen
\usepackage[section]{placeins} %\FloatBarrier Verhindert das Abbildungen in nachfolgenden Kapitel rutschen
\makeatletter\AtBeginDocument{\expandafter\renewcommand\expandafter\subsection\expandafter{\expandafter\@fb@secFB\subsection}}\makeatother %FloatBarrier für Subsection
\usepackage{subcaption} % Bilder nebeneinander
\captionsetup[subfigure]{list=true, labelfont=bf, position=top}
\usepackage{wrapfig} % Bilder von TExt umgeben


% ============= Programmierung (Visualisierung) =============

\if 1\mintedinstall
	\usepackage[cache=false]{minted} %Highlighting von Code, Benötigt Python und Pygments hinterlegt in den Umgebungsvariablen (pip install Pygments) ,cmd Eingabe->pdflatex -synctex=1 -interaction=nonstopmode -shell-escape %.tex
	\usemintedstyle{tango, xleftmargin=20pt, linenos, fontsize=\small, breaklines=true, tabsize=4} %borland, bw, tango, frame=leftline
\fi


\usepackage{listings}
%\lstset{
%	language=python,
%	otherkeywords={1, 2, 3, 4, 5, 6, 7, 8 ,9 , 0, -, =, +, [, ], (, ), \{, \}, :, *, !},
%}

\lstdefinestyle{UniStStyle}{
%	backgroundcolor=\color{uniSlightgrey},   
	commentstyle=\itshape\color{uniSlightblue}\lstset{columns=fullflexible},
	keywordstyle=\color{uniSdarkblue},
	numberstyle=\tiny\color{uniSlightgrey},
	stringstyle=\color{uniSgreyblue},
	basicstyle=\ttfamily\small,
	xleftmargin=20pt,
%	xrightmargin=3pt,
	breakatwhitespace=false,         
	breaklines=true,                 
	captionpos=b,                    
	keepspaces=true,                 
	numbers=left,                    
	numbersep=10pt,                  
	showspaces=false,                
	showstringspaces=false,
	showtabs=false,                  
	tabsize=4
}

\lstset{style=UniStStyle}


% ============= Tabelleneinstellungen =============

\usepackage{multirow}						% ermoeglicht mehrere Zeilen in einer Tabellenzeile
\usepackage{multicol} 						% mehre Spalten auf eine Seite
\usepackage{longtable, tabu}				% alternatives Tabellenpackage inkl. longtabu und Farben
\usepackage{float}							% um [H] bei float Objekten verwenden zu können
\usepackage{booktabs}						% um \toprule, \midrule und \bottomrule verwenden zu können
\usepackage{tabularx}
\usepackage{tabulary}
\usepackage{csvsimple} %Excel Improtieren als CSV 
\usepackage{siunitx} %Wissenschaftliche Angabe der Einheiten \num{number}
%für Excel to Latex Add-on
\usepackage{colortbl} %Farbige Tabellen
\usepackage{rotating} %Rotieren von Tabellen
\usepackage{bigstrut} %Für Tabellen mit Vertikalen Linien


% ============= Standart Einstellungen der Pfade =============

\graphicspath{{./Images/}{./Images/Titelseite/}} %Bilderordner


% ============= Schriftarten =============

\ifluatex
%% ============= Lua-LaTeX =============
	\usepackage[no-math]{fontspec} 
	\setmainfont{UniversforUniS}[
	Extension      = .ttf,
	Path           = fonts/,
	UprightFont    = *55Rm-Regular,
	ItalicFont     = *45LtObl-Rg,
	BoldFont       = *65Bd-Regular,
	BoldItalicFont = *45LtObl-Rg,
	]
	\setsansfont{UniversforUniS}[
	Extension      = .ttf,
	Path           = fonts/,
	UprightFont    = *55Rm-Regular,
	ItalicFont     = *45LtObl-Rg,
	BoldFont       = *65Bd-Regular,
	BoldItalicFont = *45LtObl-Rg,
	]
	%\setmonofont{}[]
	\addtokomafont{disposition}{\rmfamily} %Überschriften selbe Schriftart wie Text
	\renewcommand\familydefault\sfdefault % Formeln in Sans-Serif
	\usepackage[italic, symbolgreek]{mathastext} % Formeln in Sans-Serif
	\renewcommand\familydefault\rmdefault % Formeln in Sans-Serif

\else
%% ============= PDF-LaTeX =============
	\usepackage[T1]{fontenc} %Richtige Trennung von Umlauten
	\usepackage[scale=0.87]{tgheros}
%	\usepackage{fourier}
	\renewcommand\familydefault\sfdefault % alles in Sans-Serif
	\usepackage[italic, symbolgreek]{mathastext} % Formeln in Sans-Serif
\fi



\usepackage{setspace} %Zeilenabstand
\setstretch{1.05} %Zeilenabstand

\setlength{\parindent}{0pt} %Wenn Absatzabstand, dann Einzug unnötig
\setlength{\parskip}{7pt} %Absatzabstand



% ============= Kopf- und Fußzeile =============

\usepackage[automark, headsepline, autooneside=false]{scrlayer-scrpage}
\automark[section]{chapter}	%Automatische Kapitelangabe in Kopfzeile
\renewcommand*{\chaptermarkformat}{} %Kapitel ohne Kapitelnummer
\renewcommand*{\sectionmarkformat}{} %Kapitel ohne Kapitelnummer
\clearpairofpagestyles %aktuelle Einetsllungen aus Kopf- und Fußzeile löschen
\setkomafont{pageheadfoot}{\small}		%Font für Kopfzeile
\setkomafont{pagefoot}{\small}  		%Font für Fußzeile
\setkomafont{pagenumber}{\small} 	%Font für Seitennummer
\KOMAoptions{headsepline = true} %Kopzeile Linie

\lehead[]{\rightmark} %Zweiseitig
\cohead[]{}
\rohead[]{\leftmark\enskip \raisebox{0.7pt}{|}\enskip \thechapter} %Darstellung Kapitel

\lefoot[\pagemark]{\pagemark}
\lofoot[]{}
\cofoot[]{}
\rofoot[\pagemark]{\pagemark} %Seitenzahl


% ============= Literatur =============

\usepackage[									
natbib=true,									% Naturwissenschaftlich
backend=biber,									% Biber statt BibTex verwenden. In TexStudio einstellen!
style=numeric,									% Stil im Verzeichnis alt. authoryear
citestyle=numeric,	                            % Stil im Fließtext
giveninits=true,										
doi=false,										% doi nicht mit angeben
isbn=false,										% isbn nicht mit angeben
sorting=none,									% Sortierung des Verzeichnis nach Autor, Titel, Jahr
% sorting=none,									% Keine Sortierung des Verzeichnisses
]{biblatex}	

\addbibresource{Literatur.bib} %Bibliographiedateien laden

\urlstyle{same}	% url Schriftart anpassen auf die des Dokuments anpassen
\usepackage{xpatch}% author in bold font in bibliography
\xpretobibmacro{author}{\mkbibbold\bgroup}{}{}
\xapptobibmacro{author}{\egroup}{}{}
\xpretobibmacro{bbx:shortauthor}{\mkbibbold\bgroup}{}{}
\xapptobibmacro{bbx:shortauthor}{\egroup}{}{}
\xpretobibmacro{bbx:editor}{\mkbibbold\bgroup}{}{}
\xapptobibmacro{bbx:editor}{\egroup}{}{}


% ============= Besondere Abkürzungen ============= 

\usepackage[figurename=Abb., % Beschriftung von Abbildung zu Abb.
tablename=Tab., % Beschriftung von Tabelle zu Tab.
labelfont=bf]{caption} %Label in Bold
\KOMAoptions{numbers = noenddot} % der Punkt hinter den Kapiteln etc wird entfernt: 1.1. wird zu 1.1


% ============= Daten Plotten ============= 

\usepackage{tikz} %Grafiken Zeichnen
\if 0\tikzextern
	\usetikzlibrary{shapes, arrows, positioning, calc, chains, scopes}
\else
	\usepackage{shellesc} % Notwendig für LuaLaTeX
	\usetikzlibrary{shapes, arrows, positioning, calc, chains, scopes, external}
	\tikzexternalize[prefix=tikz/,optimize command away=\includepdf] % activate and define tikz/ as cache folder
\fi

\usepackage{pgfplots}  %Kurven, Graphen
\usepgfplotslibrary{dateplot, fillbetween, groupplots}
\pgfkeys{/pgf/number format/.cd,fixed, use comma}
\pgfplotsset{
	compat=newest,
	every axis label/.append style={font={\rmfamily}},
}
\AtBeginEnvironment{tikzpicture}{\footnotesize} %Schriftart Tabelle \small, \footnotesize

\usepgfplotslibrary{groupplots}
\usepgfplotslibrary{dateplot}

\newlength{\figureheight} %Eigene Vatiable für Längen
\newlength{\figurewidth} %Eigene Vatiable für Längen


% ============= Testing ============= 

\usepackage{blindtext}	% Beispiel Texte einfügen per \blindtext	
